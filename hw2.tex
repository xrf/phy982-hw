\documentclass[fleqn, 12pt]{article}
% Packages
\usepackage[0.9, font.kp, swapVarGreek]{sigilz}
\usepackage{hyperref}
\usepackage{url}

\usepackage{listings}
\usepackage{color}

\definecolor{dkgreen}{rgb}{0,0.6,0}
\definecolor{gray}{rgb}{0.5,0.5,0.5}
\definecolor{mauve}{rgb}{0.58,0,0.82}
\lstset{frame=tb,
  aboveskip=3mm,
  belowskip=3mm,
  showstringspaces=false,
  columns=flexible,
  basicstyle={\small\ttfamily},
  keywordstyle=\color{blue},
  commentstyle=\color{dkgreen},
  stringstyle=\color{mauve},
  breaklines=true,
  breakatwhitespace=true,
  tabsize=3
}

% Header
\setlength{\headheight}{15pt}
\pagestyle{fancy}
\lhead{Adam B Jones and Fei Yuan}
\chead{PHY982}
\marginparsep=2cm

% Environments
\definecolor{correctioncolor}{rgb}{.7, .2, 0}
\provideenvironment{correction}{\begingroup\color{correctioncolor}}{\endgroup}
\provideenvironment{valigntop}%
{\begin{minipage}[t]{\textwidth}\vspace{0pt}}%
{\vspace{0pt}\end{minipage}}

% Commands
\providecommand{\Dom}{\operatorname{Dom}}
\providecommand{\Roots}{\operatorname{Roots}}
\providecommand{\pvint}{\fint}
\providecommand{\integral}{\mathop{\textstyle\int}}
\providecommand{\ointegral}{\mathop{\textstyle\oint}}
\def\oint{\ointctrclockwise}

% note: \textcolor for fg; \colorbox for bg
\definecolor{rescolor}{rgb}{.8, .9, 1}
\newcommand{\resmath}[1]{\colorbox{rescolor}{\ensuremath{\displaystyle
      #1}}}

\rhead{HW2}
\begin{document}

\subsection*{Part I}

\begin{enumerate}

\item We choose ${}^{60}_{28} \mathrm{Ni}$ as the target nucleus.  The
  entrance partition consists of the ground states of the nuclei:
  \begin{itemize}
  \item For ${}^{60}_{28} \mathrm{Ni}$, the state is $0^+$.
  \item For ${}^{1}_{1} \mathrm{p}$ or ${}^{1}_{0} \mathrm{n}$, the state is
    $1/2^+$.
  \end{itemize}

\item Group directory: \verb|/projects/phy982/Group_AJ_FY|

\item \verb|projects/phy982/Group_AJ_FY/Ni60-p-elastic-low.in| contains:

  \begin{lstlisting}
Fresco;
NAMELIST
 &FRESCO hcm=0.1 rmatch=60
         jtmin=0.0 jtmax=50.0  absend=0.001
         thmin=0.0 thmax=180.0 thinc=1.0
         chans=1 smats=2 xstabl=1
         elab(1:1)=8.2 /
 &PARTITION namep='p' massp=1 zp=1
            namet='Ni60' masst=60 zt=28
            qval=0 nex=1  /
 &STATES jp=0.5 bandp=1 ep=0.0
         jt=0 bandt=1 et=0.0
         cpot=1 /
 &partition /
 &POT ap=1 at=60 kp=1 rc=1.25 type=0 /
 &POT kp=1 p1=53.5 p2=1.25 p3=0.65 type=1 /
 &POT kp=1 p4=13.5 p5=1.25 p6=0.47 type=2 /
 &POT kp=1 p1=7.5 p2=1.25 p3=0.47 type=3 /
 &pot /
 &overlap /
 &coupling /
  \end{lstlisting}

\item Without the nuclear interaction, there would be only Coulomb
  interaction.

  \begin{itemize}
  \item For point-like targets, the angular distribution would be
    that of Rutherford scattering.
  \item For finite-size targets, the angular distribution would be the product
    of Rutherford scattering with the Coulomb form factor of the target, which
    is the Fourier transform of the charge distribution into momentum space.
  \end{itemize}

\item We chose to use optical model parameters from RIPL-3 as suggested by the
  instructions.  Due to the large difference in energy, we needed to use
  different optical models for each of the 4 cases:
  $(\mathrm p, \mathrm n) \times (E_{\text{low}}, E_{\text{high}})$.

  These are the projectiles and energies we used:
  \begin{itemize}
  \item p, $8.2\,\mathrm{MeV}$
  \item p, $55.0\,\mathrm{MeV}$
  \item n, $5.0\,\mathrm{MeV}$
  \item n, $24.0\,\mathrm{MeV}$
  \end{itemize}

  We picked general optical models -- i.e.\ ones that are not specifically
  optimized for Ni-60.

  \begin{figure}
    \centering
    \includegraphics[width=\textwidth]{hw2-1.pdf}
    \caption{Comparison between theoretical and experimental differential
      cross sections.  Shown on top of each subplot are the energies and
      projectile respectively.  The cross sections for proton scattering are
      relative to the (point-particle) Rutherford cross section.}
    \label{fig:compare}
  \end{figure}

  The ones we used were (with their IDs):
  \begin{itemize}
  \item p, low-energy:
    (4100) F.\ G.\ Perey, \textit{Phys.\ Rev.}\ \textbf{131}, p.\ 745, 1963.
    \url{https://www-nds.iaea.org/cgi-bin/ripl_om_param.pl?Z=28&A=60&ID=4100&E1=8.2&E2=8.2}
  \item p, high-energy:
    (4102) J.\ J.\ H.\ Menet, E.\ E.\ Gross, J.\ J.\ Malanify, and A.\ Zucker,
    \textit{Phys.\ Rev.}\ \textbf{C4}, p.\ 1114, 1971.
    \url{https://www-nds.iaea.org/cgi-bin/ripl_om_param.pl?Z=28&A=60&ID=4102&E1=55&E2=55}
  \item n, low-energy:
    (401) D.\ Wilmore and P.\ E.\ Hodgson,
    \textit{Nucl.\ Phys.}\ \textbf{55}, p.\ 673, 1964.
    \url{https://www-nds.iaea.org/cgi-bin/ripl_om_param.pl?Z=28&A=60&ID=401&E1=5&E2=5}
  \item n, high-energy:
    (100) F.\ D.\ Becchetti, Jr.\ and G.\ W.\ Greenlees,
    \textit{Phys.\ Rev.}\ \textbf{182}, p.\ 1190, 1969.
    \url{https://www-nds.iaea.org/cgi-bin/ripl_om_param.pl?Z=28&A=60&ID=100&E1=24&E2=24}
  \end{itemize}

  In every calculation, to ensure convergence, we double $J_{\text{max}}$ and
  also $R_{\text{match}}$ (separately) and check that no noticeable
  differences appear in the results.

  When we compared the theoretical results with the experimental data, we find
  that the match is rather poor.  See Figure \ref{fig:compare}.

\item With the imaginary potential, we find that the S-matrix amplitude does
  not equal to one in many of the partial waves (especially those with lower
  orbital angular momentum).  Without it, the S-matrix amplitude is uniformly
  one in every partial wave.  We plot the S-matrix values in order of their
  output from Fresco in Figure \ref{fig:s-matrix}.

  \begin{figure}
    \centering
    \includegraphics[width=\textwidth]{hw2-p60Ni-OMP108-Smat.pdf}
    \caption{S-matrix output from Fresco.}
    \label{fig:s-matrix}
  \end{figure}

  We also note that for the higher energies the S-matrix appears to be lower,
  indicating that reactions are more likely to occur at the higher energies.

\item To see the effect of the radius, we increased every radial parameters
  within the input file by $50\%$ (simultaneously).  We find that this noticeably
  increases the number of oscillations in the differential cross section along
  the angle axis.  This is expected since the cross section is in effect a
  diffraction pattern in momentum space, which scales inversely with distance.
  See Figure \ref{fig:radius}

  \begin{figure}
    \centering
    \includegraphics[width=\textwidth]{hw2-1-large-radius.pdf}
    \caption{Effect of large radius parameters.}
    \label{fig:radius}
  \end{figure}

\item As noted earlier, the general optical model describes the data rather
  poorly.  The oscillations do not line up properly, and the magnitudes are
  off a bit as well.  Though, it is of the right order of magnitude, at least.

\item The calculations are once again repeated for the neutrons.  They are
  also in Figure \ref{fig:compare}.

  We find that the neutron cross section calculations are extremely poor: the
  magnitude is about 1 to 2 orders of magnitude off.  The only thing that it
  (kind of) gets right is the decreasing trend as the angle increases.

\item Total cross sections for neutron scattering:

  \begin{itemize}
  \item low energy ($5\,\mathrm{MeV}$):
    \begin{itemize}
    \item total: $192.6\,\mathrm{mb}$
    \item absorption: $101.1\,\mathrm{mb}$
    \end{itemize}

  \item high energy ($24\,\mathrm{MeV}$):
    \begin{itemize}
    \item total: $89.3\,\mathrm{mb}$
    \item absorption: $50.0\,\mathrm{mb}$
    \end{itemize}
  \end{itemize}

\end{enumerate}

\subsection*{Part II}

\begin{enumerate}

\item We consider the case in Part I and use the data from the NNCD database.
  In particular, we used the following in our plots earlier:

  \begin{itemize}
  \item p:
    \begin{itemize}
    \item $8.2\,\mathrm{MeV}$:
      (O0446003) S.\ Kobayashi, K.\ Matsuda, Y.\ Nagahara, Y.\ Oda, and N.\ Yamamuro.,
      \textit{J.\ of Physical Society of Japan} \textbf{15}, p.\ 1151, 1960.
      \url{http://www.nndc.bnl.gov/exfor/servlet/X4sGetReacTabl?reqx=20613&subID=240446003}
    \item $55.0\,\mathrm{MeV}$:
      (E1948002) M.\ Koike, K.\ Matsuda, I.\ Nonaka, Y.\ Saji, K.\ Yagi,
      H.\ Ejiri, Y.\ Ishizaki, Y.\ Nakajima, and E.\ Tanaka.,
      \textit{J.\ of Physical Society of Japan} \textbf{21} (11), p.\ 2103, 1966.
      \url{http://www.nndc.bnl.gov/exfor/servlet/X4sGetReacTabl?reqx=20613&subID=141948002}
    \end{itemize}
  \item n:
    \begin{itemize}
    \item $5.0\,\mathrm{MeV}$
      (40572003) I.\ A.\ Korzh, V.\ P.\ Lunev, V.\ A.\ Mishchenko,
      E.\ N.\ Mozhzhukhin, M.\ V.\ Pasechnik, and N.\ M.\ Pravdivyy,
      \textit{5th All-Union Conf.\ on Neutron Phys.}\ \textbf{1}, p.\ 314, 15-19 Sep 1980.
      \url{http://www.nndc.bnl.gov/exfor/servlet/X4sGetReacTabl?reqx=20823&subID=40572003}
    \item $24.0\,\mathrm{MeV}$
      (10953004) Y.\ Yamanouti, J.\ Rapaport, S.\ M.\ Grimes, V.\ Kulkarni,
      R.\ W.\ Finlay, D.\ Bainum, P.\ Grabmayr, and G.\ Randers-Pehrson,
      \textit{Conf.\ on Nucl.\ Cross Sections} p.\ 146, F.\ Techn., Knoxville, 1979.
      \url{http://www.nndc.bnl.gov/exfor/servlet/X4sGetReacTabl?reqx=20823&subID=10953004}
    \end{itemize}
  \end{itemize}

  For the rest of this report, we focus on the low-energy proton scattering case.

\item We start by fitting the $V$, $r$, and $a$ of the real volume potential,
  using the global potential parameters as the initial guess:
  $53.5\,\mathrm{MeV}$, $1.25\,\mathrm{fm}$ and $0.65\,\mathrm{fm}$
  respectively.

  \begin{figure}
    \centering
    \includegraphics[width=\textwidth]{hw2-Ni60-p-elastic-low-vol-re.pdf}
    \caption{Comparison of fit against data (real volume term).}
    \label{fig:low-vol-re}
  \end{figure}

  The $V$, by itself, does not improve the fit by much.  So we start by
  allowing $a$ to change and the minimum was found to be $0.55\,\mathrm{fm}$.

  Then we allow $r$ to change.  This causes $a$ to decrease even further to
  $0.29\,\mathrm{fm}$, which seems rather unphysical.  The $r$ is optimized to
  be $1.39\,\mathrm{fm}$.

  Finally we allow all 3 parameters to change, producing the final result of
  $84.6\,\mathrm{MeV}$, $1.03\,\mathrm{fm}$ and $0.59\,\mathrm{fm}$
  respectively.  The depth is quite a bit larger and the radius is noticeably
  smaller, while the diffusivity is slighter lower than usual.  The fit is
  shown in Figure \ref{fig:low-vol-re}.

  This fit, while noticeably better than in Part I (peaks and troughs are
  decently aligned; original fit had $\chi^2 = 297$), is far from perfect.
  $\chi^2 = 45$ for this fit, and the region near low angles is still poorly
  fitted.

\item We tweaked the initial values of the three parameters by $10\%$
  separately and refitted in the same manner.  Here is how they changed the
  fit results:
  \begin{itemize}
  \item changing $V$: fitted $V$ becomes $83.2\,\mathrm{MeV}$ instead.
  \item changing $r$: the values remain within $0.1$ of the original (in their
    respective units).
  \item changing $a$: fitted $V$ becomes $68.5\,\mathrm{MeV}$, $r$ becomes
    $0.75\,\mathrm{fm}$, and $a$ beccomes $0.20\,\mathrm{fm}$.
  \end{itemize}
  From this we see that $a$ is very sensitive to the initial value, while the
  other two are not very sensitive (with $r$ being the least sensitive).

\item (Note: there is no volume imaginary component for this particular
  model.)  Now we do the same for the surface imaginary term, also using the
  global potential as initial values.

  \begin{figure}
    \centering
    \includegraphics[width=\textwidth]{hw2-Ni60-p-elastic-low-surf-im.pdf}
    \caption{Comparison of fit against data (imaginary surface term).}
    \label{fig:low-surf-im}
  \end{figure}

  \begin{figure}
    \centering
    \includegraphics[width=\textwidth]{hw2-Ni60-p-elastic-low-both.pdf}
    \caption{Comparison of fit against data (both volume and surface term).}
    \label{fig:low-both}
  \end{figure}

  In this case, the fit in Figure \ref{fig:low-surf-im} is not as good as the
  one for the real volume potential.  The fitted parameters are:
  $W = 22.7\,\mathrm{MeV}$, $r = 1.3\,\mathrm{fm}$, and
  $a = 0.55\,\mathrm{fm}$.  We find that $\chi^2 = 113$, worse than before.

  We also tested the sensitivity of these parameters with the same technique
  as before, and found that the final parameters did not change very much for
  any of the 3 parameters (worst was about $0.5\,\mathrm{MeV}$ for $W$).  With
  this, we conclude that these 3 parameters, while they do not improve the fit
  as much, are more reliably pinned than the volume parameters.

  If we fit with both volume and surface terms, we obtain the result in Figure
  \ref{fig:low-both}.  The parameters are: $V_{\text v} = 89.2\,\mathrm{MeV}$,
  $r_{\text v} = 1.0\,\mathrm{fm}$, $a_{\text v} = 0.57\,\mathrm{fm}$,
  $W_{\text s} = 25.2\,\mathrm{MeV}$, $r_{\text s} = 1.2\,\mathrm{fm}$, and
  $a_{\text s} = 0.26\,\mathrm{fm}$.  We find that $\chi^2 = 31$, somewhat
  better than with only volume.

\item Finally, we repeat this with the real spin-orbit term, as plotted in
  \ref{fig:low-spinorb}.  The fit has been improved slightly, but the small
  angle region is still way off.

  \begin{figure}
    \centering
    \includegraphics[width=\textwidth]{hw2-Ni60-p-elastic-low-spinorb.pdf}
    \caption{Comparison of fit against data (all 3 terms).}
    \label{fig:low-spinorb}
  \end{figure}

  The fitted parameters are:
  \begin{align*}
    V_{\text v} = 95.2\,\mathrm{MeV} \\
    r_{\text v} = 1.04\,\mathrm{fm} \\
    a_{\text v} = 0.36\,\mathrm{fm} \\
    W_{\text s} = 8.8\,\mathrm{MeV} \\
    r_{\text s} = 0.99\,\mathrm{fm} \\
    a_{\text s} = 0.43\,\mathrm{fm} \\
    V_{\text o} = 11.7\,\mathrm{MeV} \\
    r_{\text o} = 1.25\,\mathrm{fm} \\
    a_{\text o} = 0.11\,\mathrm{fm}
  \end{align*}
  We find that $\chi^2 = 24$, a small but noticeable improvement.  In our
  attempts we found that $a_{\text o}$ was adjusted to an unphysical negative
  value, thus we constrained its value to be positive.  Despite this,
  $0.11\,\mathrm{fm}$ is still frighteningly low, indicating that the fit
  needs further manual adjustments from an expert who actually knows what they
  are doing.

\end{enumerate}

\newpage

\subsection*{Corrections}

\subsection*{General models}

\begin{figure}
  \centering
  \includegraphics[width=\textwidth]{hw2-1.pdf}
  \caption{Comparison between theoretical and experimental differential
    cross sections.  Shown on top of each subplot are the energies and
    projectile respectively.  The cross sections for proton scattering are
    relative to the (point-particle) Rutherford cross section.}
  \label{fig:compare2}
\end{figure}

In Figure \ref{fig:compare2} we show the comparison between the optical
potential models and the experimental data.  The optical potentials are
general: they are fitted over a wide range of nuclei, so there is a
significant discrepancy between them.

\subsection*{Fitting}

\subsubsection*{Neutron scattering at $5\,\mathrm{MeV}$}

\begin{figure}
  \centering
  \includegraphics[width=\textwidth]{hw2-Ni60-n-elastic-low-spinorb.pdf}
  \caption{Fit of neutron scattering with \textsuperscript{60}Ni at
    $5\,\mathrm{MeV}$.}
  \label{fig:n-low-spinorb}
\end{figure}

The fit (Figure \ref{fig:n-low-spinorb}) seems to be rather indifferent to
the value of $a_{\text v}$ in the volume term, so we left the parameter fixed
at $0.65\,\mathrm{fm}$.  All the other parameters are minimized according to
sfresco:
\begin{align*}
  V_{\text v} &=    54(1)\,\mathrm{MeV} \\
  r_{\text v} &=  1.06(2)\,\mathrm{fm} \\
  W_{\text s} &=     3(2)\,\mathrm{MeV} \\
  r_{\text s} &=  1.01(7)\,\mathrm{fm} \\
  a_{\text s} &= 0.62(24)\,\mathrm{fm} \\
  V_{\text o} &=    92(3)\,\mathrm{MeV} \\
  r_{\text o} &= 0.869(4)\,\mathrm{fm} \\
  a_{\text o} &=  0.57(5)\,\mathrm{fm}
\end{align*}
We note that the radii seem a bit lower than normal, and the spinorbit term
seems unusually large.  The surface term has large uncertainities, indicating
that it is poorly determined by the data.  In this fit, $\chi^2 = 0.37$, which
is a bit low (probably due to the large error bars).

\subsubsection*{Neutron scattering at $24\,\mathrm{MeV}$}

\begin{figure}
  \centering
  \includegraphics[width=\textwidth]{hw2-Ni60-n-elastic-high-spinorb.pdf}
  \caption{Fit of neutron scattering with \textsuperscript{60}Ni at
    $24\,\mathrm{MeV}$.}
  \label{fig:n-high-spinorb}
\end{figure}

We could not determine suitable values for $a_{\text s}$ or $a_{\text o}$ as
freeing them always caused severe divergence and unphysical parameters,
therefore we left them at the usual $0.65\,\mathrm{fm}$.  The fit results are
shown in Figure \ref{fig:n-high-spinorb}.  All the other parameters are
minimized according to sfresco ($\chi^2 = 6.8$):
\begin{align*}
  V_{\text v} &= 43.3(5)\,\mathrm{MeV} \\
  r_{\text v} &= 0.986(8)\,\mathrm{fm} \\
  a_{\text v} &= 0.67(1)\,\mathrm{fm} \\
  W_{\text s} &= 2.9(2)\,\mathrm{MeV} \\
  r_{\text s} &= 0.98(3)\,\mathrm{fm} \\
  V_{\text o} &= 3.1(5)\,\mathrm{MeV} \\
  r_{\text o} &= 0.94(4)\,\mathrm{fm} \\
\end{align*}
The tail of this curve (at larger angles) seems rather poorly fit.  These
parameters have the same issue of low radii similar to the previous fit.

\end{document}
