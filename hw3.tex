\documentclass[fleqn, 12pt]{article}
% Packages
\usepackage[0.9, font.kp, swapVarGreek]{sigilz}
\usepackage{hyperref}
\usepackage{url}

\usepackage{listings}
\usepackage{color}

\definecolor{dkgreen}{rgb}{0,0.6,0}
\definecolor{gray}{rgb}{0.5,0.5,0.5}
\definecolor{mauve}{rgb}{0.58,0,0.82}
\lstset{frame=tb,
  aboveskip=3mm,
  belowskip=3mm,
  showstringspaces=false,
  columns=flexible,
  basicstyle={\small\ttfamily},
  keywordstyle=\color{blue},
  commentstyle=\color{dkgreen},
  stringstyle=\color{mauve},
  breaklines=true,
  breakatwhitespace=true,
  tabsize=3
}

% Header
\setlength{\headheight}{15pt}
\pagestyle{fancy}
\lhead{Adam B Jones and Fei Yuan}
\chead{PHY982}
\marginparsep=2cm

% Environments
\definecolor{correctioncolor}{rgb}{.7, .2, 0}
\provideenvironment{correction}{\begingroup\color{correctioncolor}}{\endgroup}
\provideenvironment{valigntop}%
{\begin{minipage}[t]{\textwidth}\vspace{0pt}}%
{\vspace{0pt}\end{minipage}}

% Commands
\providecommand{\Dom}{\operatorname{Dom}}
\providecommand{\Roots}{\operatorname{Roots}}
\providecommand{\pvint}{\fint}
\providecommand{\integral}{\mathop{\textstyle\int}}
\providecommand{\ointegral}{\mathop{\textstyle\oint}}
\def\oint{\ointctrclockwise}

% note: \textcolor for fg; \colorbox for bg
\definecolor{rescolor}{rgb}{.8, .9, 1}
\newcommand{\resmath}[1]{\colorbox{rescolor}{\ensuremath{\displaystyle
      #1}}}

\rhead{HW3}
\begin{document}

\begin{enumerate}

\item %1

\item Assume we have a short-ranged, spherically symmetric potential $V(r)$.
  The incident beam is a plane wave of momentum of $k$ originating from the
  negative z-axis:
  \begin{align*}
    \psi_{\text i}(\bm r) = \E^{\I k z}
  \end{align*}
  The scattered wave has the asymptotic form
  \begin{align*}
    \psi_{\text f}(\bm r) = f(\theta) \frac{\E^{\I k r}}{r}
  \end{align*}
  where we have omitted the azimuth $\phi$ due to symmetry.

  The Schr\"odinger equation of this problem is:
  \begin{align*}
    \left(-\frac{\hbar^2}{2 m} \nabla^2 + V(r)\right) \psi(\bm r)
    = E \psi(\bm r)
  \end{align*}
  In spherical coordinates, the Laplacian is given by:
  \begin{align*}
  -\nabla^2 = \frac{1}{r^2} (\hat \ell^2 - \frac{\D}{\D r} r^2 \frac{\D}{\D r})
  \end{align*}
  where $\hat{\bm \ell}$ is the orbital angular momentum operator, whose
  eigenvalues are of the form $\ell (\ell + 1)$.  Due to azimuthal symmetry,
  the only possible angular eigenstates are the ones with $m = 0$, which are
  simply the Legendre polynomials composed with cosine: $P_\ell(\cos \theta)$.

  Hence, we shall expand the incoming wave as a sum of Legendre polynomials:
  \begin{align*}
    \E^{\I k z} =
    \sum_{\ell = 0}^\infty a_\ell P_\ell(\cos \theta)
  \end{align*}
  To do this, we use the orthogonality relations
  \begin{align*}
     \int_0^\PI P_\ell(\cos \theta) P_{\ell'}(\cos \theta) \sin \theta \D \theta
    = \frac{2}{2 \ell + 1} \Kroneckerdelta_{\ell \ell'}
  \end{align*}
  to obtain the coefficients:
  \begin{align*}
     \frac{2}{2 \ell + 1} a_\ell
    = \int_0^\PI P_\ell(\cos \theta) \E^{\I k r \cos \theta} \sin \theta \D \theta
  \end{align*}
  According to NIST 10.54.3 (\url{http://dlmf.nist.gov/10.54#E2}), this
  integral is proportional to the spherical Bessel function:
  \begin{align*}
    \dots = \frac{2}{(-\I)^\ell} j_\ell(k r)
  \end{align*}
  Thus,
  \begin{align*}
    \E^{\I k z} =
    \sum_{\ell = 0}^\infty (2 \ell + 1) \I^\ell j_\ell(k r) P_\ell(\cos \theta)
  \end{align*}
  The spherical Bessel functions $j_\ell$ (regular) and $y_\ell$ (irregular)
  are related to the Riccati--Hankel functions $\zeta_\ell$ (incoming) and
  $\xi_\ell$ (outgoing) by:
  \begin{align*}
    &\zeta_\ell(s) = s (j_\ell(s) - \I y_\ell(s)) \\
    &\xi_\ell(s) = s (j_\ell(s) + \I y_\ell(s))
  \end{align*}
  Therefore, the plane waves may also be written as:
  \begin{align*}
    \E^{\I k z} =
    \sum_{\ell = 0}^\infty (2 \ell + 1) \I^\ell
    \frac{1}{2 k r} (\zeta_\ell(k r) + \xi_\ell(k r)) P_\ell(\cos \theta)
  \end{align*}
  In this form, the partial wave S-matrix element $S_\ell$ is defined as the
  coefficient of the outgoing wave in the asymptotic wave function:
  \begin{align*}
    \psi(\bm r) \sim \sum_{\ell = 0}^\infty (2 \ell + 1) \I^\ell
    \frac{1}{2 k r} (\zeta_\ell(k r) + S_\ell \xi_\ell(k r)) P_\ell(\cos \theta)
  \end{align*}
  The incoming wave must have the same coefficient as that of the plane wave
  due to the boundary conditions.

  The scattered wave $\psi_{\text f}$ is the difference between the full wave
  function $\psi$ and the incident plane wave $\psi_{\text i}$, hence its
  asymptotic form is also their respective difference:
  \begin{align*}
    \psi_{\text f}(\bm r) \sim \sum_{\ell = 0}^\infty (2 \ell + 1) \I^\ell
    \frac{1}{2 k r} (S_\ell - 1) \xi_\ell(k r) P_\ell(\cos \theta)
  \end{align*}
  We can factor out the spherical wave
  \begin{align*}
    \psi_{\text f}(\bm r) \sim \frac{\E^{\I k r}}{r}
    \sum_{\ell = 0}^\infty (2 \ell + 1) \I^\ell
    \frac{1}{2 k} (S_\ell - 1) \frac{\xi_\ell(k r)}{\E^{\I k r}} P_\ell(\cos \theta)
  \end{align*}
  and use the asymptotic behavior of the Riccati--Hankel functions
  \begin{align*}
    \xi_\ell(s) \sim \I^{-(\ell + 1)} \E^{\I s}
  \end{align*}
  to simplify the result for large $r$:
  \begin{align*}
    \psi_{\text f}(\bm r)
    &\sim \frac{\E^{\I k r}}{r}
    \sum_{\ell = 0}^\infty (2 \ell + 1) \I^\ell
    \frac{1}{2 k} (S_\ell - 1) \I^{-(\ell + 1)} P_\ell(\cos \theta) \\
    &=\frac{\E^{\I k r}}{r}
    \frac{1}{2 \I k} \sum_{\ell = 0}^\infty (2 \ell + 1)
    P_\ell(\cos \theta) (S_\ell - 1)
  \end{align*}
  From this, we may read off the scattering amplitude $f$:
  \begin{align*}
    f(\theta) =
    \frac{1}{2 \I k} \sum_{\ell = 0}^\infty (2 \ell + 1)
    P_\ell(\cos \theta) (S_\ell - 1)
  \end{align*}

\item %3

\item Assuming a phase shift around a resonance with the usual shape:
  \begin{align*}
    \delta(E) = \delta_{\text{bg}}(E) +
    \operatorname{atan2}\left(\frac{\Gamma}{2}, E_{\text r} - E\right)
  \end{align*}
  The S-matrix element can be recovered by exponentiation:
  \begin{align*}
    S(E)
    &= \left(\E^{\I \delta(E)}\right)^2 \\
    &= \left(\E^{\I \delta_{\text{bg}}(E)}
      \E^{\I \operatorname{atan2}(\Gamma / 2, E_{\text r} - E)}\right)^2
  \end{align*}
  The arguments of $\operatorname{atan2}$ may be interpreted as the phase of a
  complex number with unit magnitude:
  \begin{align*}
    \frac{\I \Gamma / 2 + E_{\text r} - E}{
    |\I \Gamma / 2 + E_{\text r} - E|}
  \end{align*}
  Allowing the exponentiation to cancel the effect of $\operatorname{atan2}$:
  \begin{align*}
    S(E)
    &= \left(\E^{\I \delta(E)}\right)^2 \\
    &= \left(\E^{\I \delta_{\text{bg}}(E)}
      \frac{\I \Gamma / 2 + E_{\text r} - E}{
      |\I \Gamma / 2 + E_{\text r} - E|}\right)^2 \\
    &= \E^{2 \I \delta_{\text{bg}}(E)}
      \frac{(\I \Gamma / 2 + E_{\text r} - E)^2}{
      |\I \Gamma / 2 + E_{\text r} - E|^2} \\
    &= \E^{2 \I \delta_{\text{bg}}(E)}
      \frac{\I \Gamma / 2 + E_{\text r} - E}{
      -\I \Gamma / 2 + E_{\text r} - E} \\
    &= \E^{2 \I \delta_{\text{bg}}(E)}
      \frac{E - E_{\text r} - \I \Gamma / 2}{
      E - E_{\text r} + \I \Gamma / 2}
  \end{align*}
  It is evident from this equation that the S-matrix element is singular when
  $E = E_{\text r} + \I \Gamma / 2$.  In particular, it is a simple pole that,
  when expanded to first order, has the general form above.

  In the complex energy plane, there exists an eigenstate of the Schr\"odinger
  equation with a complex energy of $E_{\text r} + \I \Gamma / 2$.  One can
  infer that it is resonance state because it has a finite lifetime:
  \begin{align*}
    \Psi(\bm r; t) \propto \E^{\I (E_{\text r} + \I \Gamma / 2) t / \hbar}
    = \E^{\I E_{\text r} t / \hbar} \E^{-\Gamma t / (2 \hbar)}
  \end{align*}

\item %5

\item %6

\item %7

\item Let
  \begin{itemize}
  \item the nucleus have a gaussian density distribution
    $\rho(r) = N \E^{-(r / R_{\text T})^2}$ with radius $R_{\text T}$, and
  \item the NN interaction is given by a Yukawa form
    $V_{\text{NN}}(r) = V_0 \E^{-\mu r} / (\mu r)$
  \end{itemize}
  We wish to find the folded potential for the effective interaction between
  the nucleon and the target.

  First we use the definition of a folded potential as a convolution between
  the density and the interaction:
  \begin{align*}
    U(s)
    &= \int_{\Real^3} \rho(|\bm r - \bm s|) V_{\text{NN}}(r) \D \bm r \\
    &= \int_{\Real^3} N \E^{-(\bm r - \bm s)^2/R_{\text T}^2} V_0 \frac{\E^{-\mu r}}{\mu r} \D \bm r
  \end{align*}
  Defining $\tilde N \equiv N R_{\text T} \sqrt\PI$ and
  $\tilde \mu \equiv \mu R_{\text T}$, we have
  \begin{align*}
    U(R_{\text T} s)
    &= \frac{\tilde N V_0}{\sqrt\PI \tilde\mu}
      \int_{\Real^3} \E^{-(\bm r - \bm s)^2} \frac{\E^{-\tilde\mu r}}{r} \D \bm r
    \\
    &= \frac{\tilde N V_0}{\sqrt\PI \tilde\mu} \int_{\Real^3} \E^{-r^2 + 2 \bm r \cdot \bm s - s^2 - \tilde\mu r} r^{-1} \D \bm r
    \\
    &= \frac{\tilde N V_0}{\sqrt\PI \tilde\mu} \int_0^\infty \int_0^\PI \E^{-r^2 + 2 r s \cos \theta - s^2 - \tilde\mu r} 2 \PI r \sin \theta \D \theta \D r
    \\
    &= \frac{\sqrt\PI \tilde N V_0}{\tilde\mu s} \int_0^\infty \E^{-r^2 - \tilde\mu r - s^2} \int_0^\PI 2 r s \E^{2 r s \cos \theta} \sin \theta \D \theta \D r
    \\
    &= \frac{\sqrt\PI \tilde N V_0}{\tilde\mu s} \int_0^\infty \E^{-r^2 - \tilde\mu r - s^2} \left.\E^{2 r s \cos \theta}\right|_{\theta=-1}^{\theta=1} \D r
    \\
    &= \frac{\sqrt\PI \tilde N V_0}{\tilde\mu s} \int_0^\infty \E^{-r^2 - \tilde\mu r - s^2} \sum_{\sigma = \pm} \sigma \E^{2 r s \sigma} \D r
    \\
    &= -\frac{\sqrt\PI \tilde N V_0}{\tilde\mu s} \sum_{\sigma = \pm} \sigma \int_0^\infty \E^{-r^2 - 2 (\tilde\mu / 2 + \sigma s) r - s^2}\D r
    \\
    &= -\frac{\PI \tilde N V_0}{2 \tilde\mu s} \E^{(\tilde\mu / 2)^2} \sum_{\sigma  = \pm} \sigma \E^{\sigma \tilde\mu s} \operatorname{erfc}\left(\frac{\tilde\mu}{2} + \sigma s\right)
  \end{align*}
where $\operatorname{erfc}$ is the complementary error function.

\end{enumerate}

\end{document}
