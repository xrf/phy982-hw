\documentclass[fleqn, 12pt]{article}
% Packages
\usepackage[0.9, font.kp, swapVarGreek]{sigilz}
\usepackage{hyperref}
\usepackage{url}

\usepackage{listings}
\usepackage{color}

\definecolor{dkgreen}{rgb}{0,0.6,0}
\definecolor{gray}{rgb}{0.5,0.5,0.5}
\definecolor{mauve}{rgb}{0.58,0,0.82}
\lstset{frame=tb,
  aboveskip=3mm,
  belowskip=3mm,
  showstringspaces=false,
  columns=flexible,
  basicstyle={\small\ttfamily},
  keywordstyle=\color{blue},
  commentstyle=\color{dkgreen},
  stringstyle=\color{mauve},
  breaklines=true,
  breakatwhitespace=true,
  tabsize=3
}

% Header
\setlength{\headheight}{15pt}
\pagestyle{fancy}
\lhead{Adam B Jones and Fei Yuan}
\chead{PHY982}
\marginparsep=2cm

% Environments
\definecolor{correctioncolor}{rgb}{.7, .2, 0}
\provideenvironment{correction}{\begingroup\color{correctioncolor}}{\endgroup}
\provideenvironment{valigntop}%
{\begin{minipage}[t]{\textwidth}\vspace{0pt}}%
{\vspace{0pt}\end{minipage}}

% Commands
\providecommand{\Dom}{\operatorname{Dom}}
\providecommand{\Roots}{\operatorname{Roots}}
\providecommand{\pvint}{\fint}
\providecommand{\integral}{\mathop{\textstyle\int}}
\providecommand{\ointegral}{\mathop{\textstyle\oint}}
\def\oint{\ointctrclockwise}

% note: \textcolor for fg; \colorbox for bg
\definecolor{rescolor}{rgb}{.8, .9, 1}
\newcommand{\resmath}[1]{\colorbox{rescolor}{\ensuremath{\displaystyle
      #1}}}

\lhead{Fei Yuan (with Adam Jones)}
\rhead{HW1}
\begin{document}

\subsection*{Setup}

Consider the \textsuperscript{11}Be halo nucleus, in which the effective
nuclear interaction between the neutron and \textsuperscript{10}Be is of a
Woods-Saxon form:
\begin{align*}
V(R) = \frac{V_0}{1 + \exp((R - R_{\text{ws}}) / a_{\text{ws}})}
\end{align*}
where
\begin{itemize}
\item $V_0 = -61.1\,\mathrm{MeV}$ is the depth of the interaction;
\item $a_{\text{ws}} = 0.65\,\mathrm{fm}$ is the diffuseness;
\item $R_{\text{ws}} = 1.2 A^{1/3}\,\mathrm{fm}$ is the radius;
\item $A = 10$ is the mass number of the core;
\item $\mu = 0.0478450 \hbar^2 / (2\,\mathrm{MeV}\,\mathrm{fm}^2)$
  is the reduced mass.
\end{itemize}

\subsection*{Answers}

\begin{enumerate}

\item Given the typical shell structure: $1\mathrm s_{1/2}$,
  $1\mathrm p_{3/2}$, $1\mathrm p_{1/2}$, $1\mathrm d_{5/2}$,
  $2\mathrm s_{1/2}$, $1\mathrm d_{3/2}$, etc., one would naively expect the
  valence neutron to be in the $1\mathrm p_{1/2}$ orbital since both
  $1\mathrm s_{1/2}$ and $1\mathrm p_{3/2}$ are completely filled.  However,
  \textsuperscript{11}Be is a halo nucleus, so this simple picture would not
  hold very well.  Likely due to the weakness of the interaction between the
  core and the valence nucleon, the spacing between major shells is lower than
  usual, causing the $2\mathrm s_{1/2}$ state to intrude into the lowest major
  shell.

  Nonetheless, one would expect that the nearby orbitals $1\mathrm p_{1/2}$
  and $1\mathrm d_{5/2}$ would not be too far off, so they are likely to be
  occupied by the valence nucleon in next few excited states, contributing to
  $1/2^-$ and $5/2^+$ respectively.  This is indeed the case for the next two
  excited states, according to the nuclear
  database:\footnote{\url{https://www.nndc.bnl.gov/chart}}
  \begin{itemize}
  \item $E(1/2^-) = 320\,\mathrm{keV}$
  \item $E(5/2^+) = 1783\,\mathrm{keV}$
  \end{itemize}

\item The radial scattering equation for the partial wave with angular
  momentum $l$ is given by:
  \begin{align*}
    u''_l(R) = \left(\frac{l (l + 1)}{R^2} +
    \frac{2 \mu}{\hbar^2} \bigl(V(R) - E\bigr)\right) u_l(R)
  \end{align*}
  To avoid clutter, we fix $l$ and thus drop the subscript from $u$.  We also
  perform the following substitutions:
  \begin{align*}
    &R \leftarrow R / a_{\text{ws}}
      \text{ (similarly for other length quantities)} \\
    &E \leftarrow E / |V_0|
      \text{ (similarly for other energy quantities)} \\
    &\mu \leftarrow \mu a_{\text{ws}}^2 |V_0| / \hbar^2
  \end{align*}
  to obtain the dimensionless equation:
  \begin{align*}
    u''(R) = \left(\frac{l (l + 1)}{R^2} +
    2 \mu \left(V(R) - E\right)\right)
    u(R)
  \end{align*}
  with the following normalized potential:
  \begin{align*}
    V(R) = \frac{\operatorname{sgn} V_0}{
    1 + \exp(R - R_{\text{ws}})}
  \end{align*}
  The notational changes shall remain in effect until the end of this answer
  (\#2).

  Due to the singularity in the second derivative at $R = 0$, we shall first
  construct an approximate power series for small $R$ before dumping the rest
  of it into the ODE solver.

  We apply the Frobenius method here.  Assume power series of the form:
  \begin{align*}
    &u_l(R) = \sum_{n = 0}^\infty a_n R^{n + \lambda} \\
    &2 \mu (V(R) - E) = \sum_{n = 0}^\infty b_n R^n
  \end{align*}
  with $a_0 \ne 0$ to avoid unnecessary ambiguity\footnote{If the first
    coefficient $a_0$ is zero, we can just redefine the series so that the
    first coefficient becomes nonzero.}  and $\lambda \ge 0$ to avoid a
  singularity in $u$ at $R = 0$.  Substituting this into the differential
  equation and grinding through the algebra, one obtains $\lambda = l + 1$ as
  well as the following recursion relation:
  \begin{align*}
    a_n n (2 l + n + 1) = \sum_{i = 0}^{n - 2} b_i a_{n - 2 - i}
  \end{align*}
  Since $a_0$ is arbitrary, we simply set it to one.\footnote{This, in effect,
    determines the coefficient of the wavefunction.}  Using this series
  (truncated to some order) we evaluate $u$ and $u'$ near the origin at
  $R = R_{\text c}$.  The results, $u_{\text c}$ and $u'_{\text c}$, are then
  used to set up the initial conditions for the integration that follows.

  \begin{figure}
    \centering
    \includegraphics[width=\textwidth]{hw1-u.pdf}
    \caption{The radial behavior of the scattering wavefunctions for
      $l \in \{0, 1, 2\}$ and $E \in \{0.1, 3.0\} \,\mathrm{MeV}$.}
    \label{fig:wavefunction}
  \end{figure}

  In this second part where $R > R_{\text c}$, we use the Runge-Kutta-Fehlberg
  (RKF45) ordinary differential equation (ODE) solver.  This method, like most
  ODE solvers, assume the standard first-order form:
  \begin{align*}
    &\bm y'(x) = \bm f(x, \bm y(x)) &
    &\text{with initial condition} &
    &\bm y(x_0) = \bm y_0
  \end{align*}
  where $\bm y$ is an $n$-dimensional solution vector and $\bm f$ is an
  $n$-dimensional vector function that computes its derivative.  Many problems
  involving differential equations can be rewritten into this form.

  The Runge-Kutta (RK) family of methods use an $m$-stage iteration to compute
  the the next step
  \begin{align*}
    \bm y \twoheadleftarrow \bm y + \bm k \bm b \Delta x
  \end{align*}
  where $\bm k$ is an $n \times m$ matrix whose columns are
  \begin{align*}
    \bm k_i = \bm f(x + c_i \Delta x, \bm y + \bm k \bm a_i \Delta x)
  \end{align*}
  What differs between the RK methods are:
  \begin{itemize}
  \item the RK matrix $\bm a$, which is $m \times m$ and strictly lower
    triangular,
  \item the weights $\bm b$: an $m$-dimensional vector, and
  \item the nodes $\bm c$: another $m$-dimensional vector,
  \end{itemize}
  all of which must be carefully chosen to produce accurate results.

  RKF45 is an adaptive solver: it uses both 4-th and 5-th order RK methods to
  estimate the error and adjust the step size during the integration process.

  Now, to use this solver, we rewrite the equation in the standard first-order
  form:
  \begin{align*}
    \begin{cases}
      u'(R) = \dot u(R) &
      u(R_{\text c}) = u_{l,{\text c}} \\
      \dot u'(R) = \left(l (l + 1) / R^2 +
        2 \mu \left(V(R) - E\right)\right)
      u(R) &
      \dot u(R_{\text c}) = \dot u_{l,{\text c}} \\
    \end{cases}
  \end{align*}
  We integrate until a $R = R_{\text e}$ at which the function becomes
  indiscernible from the asymptotic solution.  At this point, we compute
  \begin{align*}
    \rho = \frac{u(R_{\text e})}{u'(R_{\text e})}
  \end{align*}
  This ratio is then used to calculate $S_l$:
  \begin{align*}
    S = \frac{H^-(R_{\text e}) - \rho H^-{}'(R_{\text e})}{
    H^+(R_{\text e}) - \rho H^+{}'(R_{\text e})}
  \end{align*}
  from which the phase shift can be derived:
  \begin{align*}
    \delta = \frac{1}{2 \I} \ln S
  \end{align*}

\item The scattering wavefunction for $l \in \{0, 1, 2\}$ and
  $E \in \{0.1, 3.0\} \,\mathrm{MeV}$ are plotted in Figure
  \ref{fig:wavefunction}.

\item Figure \ref{fig:phase-shift} shows how the phase shift $\delta_l$ varies
  as a function of energy for $l \in \{0, 1, 2\}$.

  \begin{figure}
    \centering
    \includegraphics[width=\textwidth]{hw1-delta.pdf}
    \caption{The energy dependence of the phase shifts for
      $l \in \{0, 1, 2\}$.}
    \label{fig:phase-shift}
  \end{figure}

  It can be seen that for $l = 0$ the phase shift is quite negative,
  indicating that there is significant attraction between the core and the
  valence neutron.

  In constrast, the $l = 1$ partial wave has a phase shift of nearly zero,
  indicating that it is neither attractive nor repulsive.

  The $l = 2$ partial wave is predominantly positive, indicating that it is
  mostly repulsive.  However, the sudden increase near $1.3\,\mathrm{MeV}$
  indicates that there is at least a resonance state.

\item There appears to be a significant resonance only for $l = 2$, as
  indicated by the large rise in phase shift.  To confirm this, the model is
  fitted to the curve as shown in \ref{fig:phase-shift} using unweighted
  Levenberg--Marquardt.  The model is defined as:
  \begin{align*}
    \delta_{\text{res}}(E) =
    \operatorname{atan2}\left(\frac{\Gamma}{2}, E_{\text r} - E\right) +
    A E + B
  \end{align*}
  The latter two terms constitute the background, assumed to be linear.

  The resonance is located at $E_{\text r} = 1.3463(1)\,\mathrm{MeV}$ and has
  a width of $\Gamma = 0.2250(2)\,\mathrm{MeV}$. The two background parameters
  are: $A = -0.0737(2)\,\mathrm{rad}\,\mathrm{MeV}^{-1}$ and
  $B = -0.0845(4)\,\mathrm{rad}$.  The uncertainties shown here are from the
  fits.

  The fit is quite good: there is no visually discernible difference between
  the actual and the model curves near the resonant region ($R^2 = 0.999996$.)

\end{enumerate}

\end{document}
